\documentclass[final,a4paper,openany,12pt]{mwbk}
\pdfminorversion=7
\usepackage{polski}
\usepackage[polish]{babel}
\usepackage[utf8]{inputenc}
\usepackage{lmodern}
\usepackage{tgtermes}
\usepackage[T1]{fontenc}
\input glyphtounicode
\pdfgentounicode=1
\usepackage{amssymb}
\usepackage{amsmath}
\usepackage{bbm}
\usepackage{tikz}
\usetikzlibrary{automata, positioning, arrows.meta}
\usepackage{minted}
\usepackage{amsthm}
\usepackage{gensymb}
\usepackage{mathrsfs}
\usepackage{biblatex}
\usepackage{csquotes}
\usepackage{graphicx}
\usepackage{float}
\addbibresource{ref.bib}

% ustawienia do wydruku dwustronnego z uwzględnieniem dodatkowego miejsca na zszycie
\setlength{\oddsidemargin}{0.46cm} %margines nieparzysty
\setlength{\evensidemargin}{-0.54cm} %margines parzysty
\setlength{\textwidth}{16cm} %szerokość tekstu na stronie
\linespread{1.1} % lekkie zwiększenie odstępu między liniami, żeby tekst nie był taki ścisły, ponieważ
% Odstęp pojedynczej interlinii nie jest komfortowy, kiedy trzeba czytać strony A4
% koniec ustawień
\fontfamily{lmr}\selectfont

\begin{document}
\newtheorem{Tw}{Twierdzenie}
\newtheorem{Def}{Definicja}
\newtheorem{Prz}{Przykład}
% instrukcja poniżej: wybór czcionki z pakietu 'lmodern' jako domyślnej
\fontfamily{lmr}\selectfont % wybór czcionki "Latin Modern Roman"
%\fontfamily{lmss}\selectfont % wybór czcionki "Latin Modern Sans Serif"

\begin{titlepage}
\vspace{-0.5cm}

\renewcommand{\arraystretch}{1.3} % zwiększamy odległość między wierszami na stronie tytułowej

\begin{center}
{\footnotesize
\begin{tabular}{c}
UNIWERSYTET KARDYNAŁA STEFANA WYSZYŃSKIEGO\\
W WARSZAWIE\\
\end{tabular}
}
\vspace{2.5cm}

{\footnotesize
\begin{tabular}{c}
WYDZIAŁ MATEMATYCZNO-PRZYRODNICZY\\
SZKOŁA NAUK ŚCISŁYCH\\
\end{tabular}
}
\vspace{2.7cm}

\renewcommand{\arraystretch}{1.5} % zwiększamy odległość między wierszami

{\normalsize
\begin{tabular}{c}
Korneliusz Pogorzelczyk\\

109204\\

matematyka\\
\end{tabular}
}

\vspace{2.3cm}

{\large
\begin{tabular}{c}
Łańcuchy Markowa
\end{tabular}
}

\end{center}
\vspace{4cm}

\hspace{6cm}
\begin{tabular}{l}
Praca licencjacka\\

Promotor:\\

dr Tomasz Rogala
\end{tabular}

\vspace{3.5cm}

{\centering

{\small
\begin{tabular}{c}
{WARSZAWA, 2025}\\
\end{tabular}
}

}

\renewcommand{\arraystretch}{1} % przywracamy domyślną odległość miedzy wierszami na następnych stronach

\end{titlepage}

\chapter{Wstęp}

\section{Motywacja i wprowadzenie}

W dzisiejszym świecie, gdzie losowość i niepewność odgrywają kluczową rolę w wielu obszarach nauki i techniki, procesy stochastyczne stanowią fundament matematycznego opisu zjawisk losowych. Spośród różnorodnych rodzajów procesów stochastycznych łańcuchy Markowa wyróżniają się szczególną elegancją matematyczną oraz szerokim wachlarzem zastosowań praktycznych.

Łańcuchy Markowa, nazwane na cześć rosyjskiego matematyka Andrieja Markowa, który wprowadził to pojęcie na początku XX wieku, charakteryzują się szczególną własnością: przyszły stan procesu zależy wyłącznie od stanu obecnego, a nie od stanów poprzednich. Ta własność (zwana brakiem pamięci lub własnością Markowa) znacząco upraszcza analizę matematyczną tych procesów, jednocześnie zachowując ich zdolność do modelowania złożonych zjawisk rzeczywistych.

Zastosowania łańcuchów Markowa są niezwykle różnorodne i obejmują takie dziedziny jak:
\begin{itemize}
    \item Finanse i ekonomia -- modelowanie rynków finansowych, ocena ryzyka kredytowego, teoria podejmowania decyzji;
    \item Biologia i medycyna -- modelowanie rozprzestrzeniania się chorób, analiza sekwencji DNA, badanie procesów ewolucyjnych;
    \item Fizyka -- opis układów cząstek, mechanika statystyczna, procesy dyfuzji;
    \item Informatyka -- algorytmy uczenia maszynowego, generowanie tekstu, kompresja danych, metoda Monte Carlo oparta na łańcuchach Markowa.
\end{itemize}

W niniejszej pracy skupimy się na łańcuchach Markowa w czasie dyskretnym, które stanowią fundamentalny i intuicyjny wariant tej klasy procesów stochastycznych. Zbadamy ich podstawowe własności teoretyczne oraz przedstawimy praktyczne zastosowanie, ilustrując teorię konkretnym przykładem implementacji.

\section{Cele i zakres pracy}

Głównym celem niniejszej pracy jest przedstawienie teorii łańcuchów Markowa w czasie dyskretnym oraz demonstracja ich praktycznego zastosowania. W szczególności, praca stawia sobie następujące cele:

\begin{enumerate}
    \item Zaprezentowanie niezbędnych podstaw rachunku prawdopodobieństwa, stanowiących fundament teoretyczny dla zrozumienia łańcuchów Markowa.
    
    \item Wprowadzenie formalnej definicji łańcuchów Markowa w czasie dyskretnym oraz szczegółowa analiza ich kluczowych własności, ze szczególnym uwzględnieniem:
    \begin{itemize}
        \item Klasyfikacji stanów (stany przejściowe, pochłaniające, okresowe i ergodyczne),
        \item Zachowania długoterminowego (rozkłady stacjonarne, ergodyczność),
        \item Czasów pierwszego przejścia i powrotu do stanów.
    \end{itemize}
    
\end{enumerate}

Zakres pracy ogranicza się do łańcuchów Markowa w czasie dyskretnym, co oznacza, że stan procesu zmienia się w dyskretnych chwilach czasowych (krokach). Główny nacisk położony zostanie na łańcuchy o skończonej przestrzeni stanów, choć w wybranych miejscach wspomnimy również o przypadkach z przeliczalnie nieskończoną przestrzenią stanów, jeśli będzie to istotne dla omawianego zastosowania.

Praca koncentruje się na podstawowych koncepcjach i własnościach łańcuchów Markowa, stanowiąc wprowadzenie do tego obszaru teorii procesów stochastycznych. Bardziej zaawansowane zagadnienia, takie jak uogólnienia do czasu ciągłego czy łańcuchy Markowa wyższego rzędu, wykraczają poza zakres niniejszej pracy.

\chapter{Wybrane zagadnienia rachunku prawdopodobieństwa}

\section{Aksjomaty przestrzeni probabilistycznej}

Zaczynamy od podania definicji przestrzeni probabilistycznej zaproponowanej prawie sto lat temu i obecnie najczęściej używanej.

\begin{Def}[Kołmogorow]
Niech dane będą niepusty zbiór $\Omega$, pewna rodzina $\mathcal{F}$ podzbiorów zbioru $\Omega$ oraz funkcja $P \colon \mathcal{F} \longrightarrow \mathbb{R}$. Trójkę $(\Omega, \mathcal{F}, P)$ nazywamy przestrzenią probabilistyczną jeśli zachodzą następujące warunki:
\begin{enumerate}
    \item $\Omega \in \mathcal{F}$,
    \item jeżeli zbiory $A_1, A_2, A_3, \dots \in \mathcal{F}$, to $\bigcup_{i=1}^{\infty} A_i \in \mathcal{F}$,
    \item jeżeli $A, B \in \mathcal{F}$, to $A \setminus B \in \mathcal{F}$,
    \item jeżeli $A \in \mathcal{F}$, to $P(A) \ge 0$,
    \item jeżeli zbiory $A_1, A_2, A_3, \dots \in \mathcal{F}$ są parami rozłączne, to:
    \[ P\left(\bigcup_{i=1}^{\infty} A_i\right) = \sum_{i=1}^{\infty} P(A_i), \]
    \item $P(\Omega) = 1$.
\end{enumerate}
\end{Def}


\begin{itemize}
    \item $\omega \in \Omega$ – zdarzenie elementarne
    \item $A \in \mathcal{F}$ – zdarzenie
    \item $\Omega \setminus A$ – zdarzenie przeciwne do $A$
    \item $\emptyset$ – zdarzenie niemożliwe
    \item $\Omega$ – zdarzenie pewne
    \item Rodzinę $\mathcal{F}$ spełniająca warunki 1 - 3 nazywamy $\sigma$-algebrą
    \item Funkcję $P$ spełniająca warunki 4 - 6 nazywamy miara probabilistyczną
    \item $P(A)$ – prawdopodobieństwo zdarzenia $A$,
\end{itemize}


\begin{Prz}
Rzut kostką sześcienną -  czy wypadnie „1” (zdarzenie A)?
\[ \Omega = \{1, 2, 3, 4, 5, 6\}, \quad \mathcal{F} = \{\emptyset, \{2, 3, 4, 5, 6\}, \{1\}, \Omega\}, \quad P(A) = \frac{1}{6}. \]
\end{Prz}

\noindent\textbf{Podstawowe własności:}
\begin{enumerate}
    \item $\emptyset \in \mathcal{F}$, bo $\emptyset = \Omega \setminus \Omega$.

    \item Jeżeli zbiory $A_1, A_2, A_3, \dots \in \mathcal{F}$, to $\bigcap_{i=1}^{\infty} A_i \in \mathcal{F}$, co wynika z praw de Morgana.

    \item $P(\emptyset) = 0$, bo $P(\emptyset) = P\left(\bigcup_{i=1}^{\infty} \emptyset\right) = \sum_{i=1}^{\infty} P(\emptyset)$.

    \item Jeżeli zbiory $A_1, A_2, A_3, \dots, A_n \in \mathcal{F}$ oraz $A_i \cap A_j = \emptyset$ dla $i \neq j$, to:
    $P\left(\bigcup_{i=1}^{n} A_i\right) = \sum_{i=1}^{n} P(A_i)$, bo $P\left(\bigcup_{i=1}^{n} A_i\right) = P\left(\bigcup_{i=1}^{\infty} A_i\right)$, gdzie $A_i = \emptyset$ dla $i > n$.

    \item Jeżeli $A$ i $B$ są takimi zdarzeniami, że $A \subset B$, to:
    \[ P(B) = P(A) + P(B \setminus A), \] bo $B = A \cup (B \setminus A)$.

    \item Dla każdego zdarzenia $A$: \[ P(\Omega\setminus A) = 1 - P(A) \]

    \item Jeżeli $A$ i $B$ są takimi zdarzeniami, że $A \subset B$, to:
    \[ P(A) \leq P(B), \]

    \item Dla dowolnych zdarzeń $A$ i $B$:
    \[ P(A \cup B) = P(A) + P(B) - P(A \cap B), \] bo $A \cup B = (A \setminus B) \cup (B \setminus A) \cup (A \cap B)$
\end{enumerate}

\noindent\textbf{Własności ciągów zdarzeń:}
\begin{enumerate}
    \item Dla dowolnych zdarzeń $A_1, A_2, A_3, \dots$:
    \[ P\left(\bigcup_{i=1}^{\infty} A_i\right) \leq \sum_{i=1}^{\infty} P(A_i), \]
    bo $\bigcup_{i=1}^{\infty} A_i = \bigcup_{i=1}^{\infty} B_i$ – suma zbiorów rozłącznych, gdzie
    $B_1 = A_1$, $B_2 = A_2 \setminus A_1$, $B_3 = A_3 \setminus (A_1 \cup A_2)$, \dots.

    \item Jeżeli $P(A_i) = 0$, $i = 1, \dots n$, $n \leq \infty$, to $P\left(\bigcup_{i=1}^{n} A_i\right) = 0$.

    \item Jeżeli $P(A_i) = 1$, $i = 1, \dots n$, $n \leq \infty$, to $P\left(\bigcap_{i=1}^{n} A_i\right) = 1$. \\
    cy wynika z praw de Morgana.
\end{enumerate}

\begin{Tw}[o ciągu zdarzeń wstępujących]
Jeżeli $A_1 \subset A_2 \subset A_3 \subset \dots$, to:
\[ \lim_{n\to\infty} P(A_n) = P\left(\bigcup_{n=1}^{\infty} A_n\right), \]
\end{Tw}
\begin{proof}
$P\left(\bigcup_{n=1}^{\infty} A_n\right) = P\left(\bigcup_{n=1}^{\infty} (A_n \setminus A_{n-1})\right) = \sum_{n=1}^{\infty} P(A_n \setminus A_{n-1}) = \lim_{N\to\infty} \sum_{n=1}^{N} P(A_n \setminus A_{n-1}) = \lim_{N\to\infty} P(A_N)$. Tutaj $A_0 = \emptyset$.
\end{proof}

\begin{Tw}[o ciągu zdarzeń zstępujących]
Jeżeli $A_1 \supset A_2 \supset A_3 \supset \dots$, to:
\[ \lim_{n\to\infty} P(A_n) = P\left(\bigcap_{n=1}^{\infty} A_n\right). \]
\end{Tw}




\printbibliography

\end{document}