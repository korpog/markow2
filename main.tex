\documentclass[final,a4paper,openany,12pt]{mwbk}
\pdfminorversion=7
\usepackage{polski}
\usepackage[polish]{babel}
\usepackage[utf8]{inputenc}
\usepackage{lmodern}
\usepackage{tgtermes}
\usepackage[T1]{fontenc}
\input glyphtounicode
\pdfgentounicode=1
\usepackage{amssymb}
\usepackage{amsmath}
\usepackage{bbm}
\usepackage{tikz}
\usetikzlibrary{automata, positioning, arrows.meta}
\usepackage{minted}
\usepackage{amsthm}
\usepackage{gensymb}
\usepackage{mathrsfs}
\usepackage{biblatex}
\usepackage{csquotes}
\usepackage{graphicx}
\usepackage{float}
\addbibresource{ref.bib}

% ustawienia do wydruku dwustronnego z uwzględnieniem dodatkowego miejsca na zszycie
\setlength{\oddsidemargin}{0.46cm} %margines nieparzysty
\setlength{\evensidemargin}{-0.54cm} %margines parzysty
\setlength{\textwidth}{16cm} %szerokość tekstu na stronie
\linespread{1.1} % lekkie zwiększenie odstępu między liniami, żeby tekst nie był taki ścisły, ponieważ
% Odstęp pojedynczej interlinii nie jest komfortowy, kiedy trzeba czytać strony A4
% koniec ustawień
\fontfamily{lmr}\selectfont

\begin{document}
\newtheorem{Tw}{Twierdzenie}
\newtheorem{Def}{Definicja}
% instrukcja poniżej: wybór czcionki z pakietu 'lmodern' jako domyślnej
\fontfamily{lmr}\selectfont % wybór czcionki "Latin Modern Roman"
%\fontfamily{lmss}\selectfont % wybór czcionki "Latin Modern Sans Serif"

\begin{titlepage}
\vspace{-0.5cm}

\renewcommand{\arraystretch}{1.3} % zwiększamy odległość między wierszami na stronie tytułowej

\begin{center}
{\footnotesize
\begin{tabular}{c}
UNIWERSYTET KARDYNAŁA STEFANA WYSZYŃSKIEGO\\
W WARSZAWIE\\
\end{tabular}
}
\vspace{2.5cm}

{\footnotesize
\begin{tabular}{c}
WYDZIAŁ MATEMATYCZNO-PRZYRODNICZY\\
SZKOŁA NAUK ŚCISŁYCH\\
\end{tabular}
}
\vspace{2.7cm}

\renewcommand{\arraystretch}{1.5} % zwiększamy odległość między wierszami

{\normalsize
\begin{tabular}{c}
Korneliusz Pogorzelczyk\\

109204\\

matematyka\\
\end{tabular}
}

\vspace{2.3cm}

{\large
\begin{tabular}{c}
Łańcuchy Markowa
\end{tabular}
}

\end{center}
\vspace{4cm}

\hspace{6cm}
\begin{tabular}{l}
Praca licencjacka\\

Promotor:\\

dr Tomasz Rogala
\end{tabular}

\vspace{3.5cm}

{\centering

{\small
\begin{tabular}{c}
{WARSZAWA, 2025}\\
\end{tabular}
}

}

\renewcommand{\arraystretch}{1} % przywracamy domyślną odległość miedzy wierszami na następnych stronach

\end{titlepage}

\chapter{Wstęp}

\section{Motywacja i wprowadzenie}

W dzisiejszym świecie, gdzie losowość i niepewność odgrywają kluczową rolę w wielu obszarach nauki i techniki, procesy stochastyczne stanowią fundament matematycznego opisu zjawisk losowych. Spośród różnorodnych rodzajów procesów stochastycznych łańcuchy Markowa wyróżniają się szczególną elegancją matematyczną oraz szerokim wachlarzem zastosowań praktycznych.

Łańcuchy Markowa, nazwane na cześć rosyjskiego matematyka Andrieja Markowa, który wprowadził to pojęcie na początku XX wieku, charakteryzują się szczególną własnością: przyszły stan procesu zależy wyłącznie od stanu obecnego, a nie od stanów poprzednich. Ta własność (zwana brakiem pamięci lub własnością Markowa) znacząco upraszcza analizę matematyczną tych procesów, jednocześnie zachowując ich zdolność do modelowania złożonych zjawisk rzeczywistych.

Zastosowania łańcuchów Markowa są niezwykle różnorodne i obejmują takie dziedziny jak:
\begin{itemize}
    \item Finanse i ekonomia -- modelowanie rynków finansowych, ocena ryzyka kredytowego, teoria podejmowania decyzji;
    \item Biologia i medycyna -- modelowanie rozprzestrzeniania się chorób, analiza sekwencji DNA, badanie procesów ewolucyjnych;
    \item Fizyka -- opis układów cząstek, mechanika statystyczna, procesy dyfuzji;
    \item Informatyka -- algorytmy uczenia maszynowego, generowanie tekstu, kompresja danych, metoda Monte Carlo oparta na łańcuchach Markowa.
\end{itemize}

W niniejszej pracy skupimy się na łańcuchach Markowa w czasie dyskretnym, które stanowią fundamentalny i intuicyjny wariant tej klasy procesów stochastycznych. Zbadamy ich podstawowe własności teoretyczne oraz przedstawimy praktyczne zastosowanie, ilustrując teorię konkretnym przykładem implementacji.

\section{Cele i zakres pracy}

Głównym celem niniejszej pracy jest przedstawienie teorii łańcuchów Markowa w czasie dyskretnym oraz demonstracja ich praktycznego zastosowania. W szczególności, praca stawia sobie następujące cele:

\begin{enumerate}
    \item Zaprezentowanie niezbędnych podstaw rachunku prawdopodobieństwa, stanowiących fundament teoretyczny dla zrozumienia łańcuchów Markowa.
    
    \item Wprowadzenie formalnej definicji łańcuchów Markowa w czasie dyskretnym oraz szczegółowa analiza ich kluczowych własności, ze szczególnym uwzględnieniem:
    \begin{itemize}
        \item Klasyfikacji stanów (stany przejściowe, pochłaniające, okresowe i ergodyczne),
        \item Zachowania długoterminowego (rozkłady stacjonarne, ergodyczność),
        \item Czasów pierwszego przejścia i powrotu do stanów.
    \end{itemize}
    
    \item Opracowanie i analiza praktycznego zastosowania łańcuchów Markowa do modelowania wybranego zjawiska wraz z implementacją obliczeniową demonstrującą kluczowe aspekty teoretyczne.
\end{enumerate}

Zakres pracy ogranicza się do łańcuchów Markowa w czasie dyskretnym, co oznacza, że stan procesu zmienia się w dyskretnych chwilach czasowych (krokach). Główny nacisk położony zostanie na łańcuchy o skończonej przestrzeni stanów, choć w wybranych miejscach wspomnimy również o przypadkach z przeliczalnie nieskończoną przestrzenią stanów, jeśli będzie to istotne dla omawianego zastosowania.

Praca koncentruje się na podstawowych koncepcjach i własnościach łańcuchów Markowa, stanowiąc wprowadzenie do tego obszaru teorii procesów stochastycznych. Bardziej zaawansowane zagadnienia, takie jak uogólnienia do czasu ciągłego czy łańcuchy Markowa wyższego rzędu, wykraczają poza zakres niniejszej pracy.

W kolejnych rozdziałach pracy systematycznie rozwijamy przedstawione zagadnienia, począwszy od niezbędnych podstaw probabilistycznych, poprzez formalną teorię łańcuchów Markowa, aż po praktyczne zastosowanie zilustrowane implementacją obliczeniową.

\printbibliography

\end{document}