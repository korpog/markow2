\documentclass{article}
\usepackage[polish]{babel}
\usepackage[utf8]{inputenc}
\usepackage{amsmath}
\usepackage{amssymb}
\usepackage{amsthm} % Do definicji i przykładów

% Definicje i przykłady
\newtheorem{Def}{Definicja}
\newtheorem{Prz}{Przykład}

\begin{document}

\section*{Zmienne Losowe i Ich Rozkłady}

\begin{Def}[Wektor losowy]
Niech $(\Omega, \mathcal{F}, P)$ będzie przestrzenią probabilistyczną.
Funkcja $X \colon \Omega \longrightarrow \mathbb{R}^n$ jest wektorem losowym, jeżeli jest ona funkcją mierzalną względem $\sigma$-algebry $\mathcal{F}$, tj.
\[
X^{-1}(B) = \{\omega \in \Omega : X(\omega) \in B\} \in \mathcal{F}
\]
dla każdego zbioru borelowskiego $B \in \mathcal{B}(\mathbb{R}^n)$.
\par\noindent
\textbf{Zmienna losowa} jest jednowymiarowym wektorem losowym ($n=1$).
\end{Def}

\begin{Def}[Rozkład wektora losowego]
Rozkład wektora losowego $X \colon \Omega \longrightarrow \mathbb{R}^n$ jest to miara probabilistyczna $P_X$ określona na $\mathcal{B}(\mathbb{R}^n)$ wzorem:
\[
P_X(B) = P(X^{-1}(B)), \quad \text{ dla } B \in \mathcal{B}(\mathbb{R}^n).
\]
Innymi słowy, $P_X(B)$ to prawdopodobieństwo, że wektor losowy $X$ przyjmie wartość ze zbioru $B$.
\end{Def}

\begin{Prz}
Rozważmy jednowymiarową zmienną losową $X$ reprezentującą sumę wyników z rzutu dwiema symetrycznymi sześciennymi kostkami do gry.
\begin{itemize}
    \item Przestrzeń zdarzeń elementarnych $\Omega$ to zbiór wszystkich możliwych par wyników $(k, l)$, gdzie $k,l \in \{1, 2, 3, 4, 5, 6\}$.
    Moc zbioru $\Omega$ wynosi $6 \times 6 = 36$. Każdy wynik $(k,l)$ ma prawdopodobieństwo $\frac{1}{36}$.
    \item Zmienna losowa $X$ jest funkcją $X(k,l) = k+l$.
    \item Możliwe wartości zmiennej losowej $X$ (sumy oczek) to liczby całkowite z przedziału $[2, 12]$.
\end{itemize}

Poniżej przedstawiono rozkład prawdopodobieństwa zmiennej losowej $X$:
\begin{align*}
P(X=2)   &= P(\{(1,1)\}) = \frac{1}{36} \\
P(X=3)   &= P(\{(1,2), (2,1)\}) = \frac{2}{36} \\
P(X=4)   &= P(\{(1,3), (2,2), (3,1)\}) = \frac{3}{36} \\
P(X=5)   &= P(\{(1,4), (2,3), (3,2), (4,1)\}) = \frac{4}{36} \\
P(X=6)   &= P(\{(1,5), (2,4), (3,3), (4,2), (5,1)\}) = \frac{5}{36} \\
P(X=7)   &= P(\{(1,6), (2,5), (3,4), (4,3), (5,2), (6,1)\}) = \frac{6}{36} \\
P(X=8)   &= P(\{(2,6), (3,5), (4,4), (5,3), (6,2)\}) = \frac{5}{36} \\
P(X=9)   &= P(\{(3,6), (4,5), (5,4), (6,3)\}) = \frac{4}{36} \\
P(X=10)  &= P(\{(4,6), (5,5), (6,4)\}) = \frac{3}{36} \\
P(X=11)  &= P(\{(5,6), (6,5)\}) = \frac{2}{36} \\
P(X=12)  &= P(\{(6,6)\}) = \frac{1}{36}
\end{align*}
Sumując wszystkie prawdopodobieństwa, otrzymujemy $\sum_{k=2}^{12} P(X=k) = \frac{36}{36} = 1$, co potwierdza, że jest to prawidłowy rozkład prawdopodobieństwa.
\end{Prz}

\end{document}